\documentclass{article}
\usepackage{amsmath}

\begin{document}
\documentclass{article}
\usepackage{amsmath}

\begin{document}

\section*{a) Conversões}

\begin{itemize}
    \item Autonomía: \(25 \, \text{km} = 25 \times 1000 \, \text{m} = 25.000 \, \text{m}\)
    \item Vmax: \(25 \, \text{km/h} = 6,94 \, \text{m/s} \left( 25 \, \text{km/h} \times \frac{1000 \, \text{m}}{1 \, \text{km}} \times \frac{1 \, \text{h}}{3600 \, \text{s}} \right)\)
    \item \(V_0: 0 \, \text{km/h} = 0 \, \text{m/s}\)
    \item \(V: 15,3 \, \text{km/h} = 4,25 \, \text{m/s} \left( 15,3 \, \text{km/h} \times \frac{1000 \, \text{m}}{1 \, \text{km}} \times \frac{1 \, \text{h}}{3600 \, \text{s}} \right)\)
    \item Tiempo: \(1,7 \, \text{segundos} = 1,7 \, \text{s}\)
\end{itemize}


\section*{b) Cálculo de la aceleración media}

Para calcular la aceleración media (\(a\)) que experimenta el patinete, podemos utilizar la fórmula de la aceleración media, que se da por:

\[
a = \frac{V - V_0}{\Delta t}
\]

donde:
\begin{itemize}
    \item \(V\) es la velocidad final,
    \item \(V_0\) es la velocidad inicial,
    \item \(\Delta t\) es el intervalo de tiempo durante el cual ha ocurrido la aceleración.
\end{itemize}

Sustituyendo los valores:

\begin{itemize}
    \item \(V_0 = 0 \, \text{m/s}\) (inicialmente en reposo),
    \item \(V = 4,25 \, \text{m/s}\) (velocidad final después de la aceleración),
    \item \(\Delta t = 1,7 \, \text{s}\) (tiempo de aceleración).
\end{itemize}

Ahora podemos calcular:

\[
a = \frac{4,25 \, \text{m/s} - 0 \, \text{m/s}}{1,7 \, \text{s}} = \frac{4,25 \, \text{m/s}}{1,7 \, \text{s}} \approx 2,5 \, \text{m/s}^2
\]

\subsection*{Justificación del Signo del Resultado}

El signo de la aceleración es positivo, lo que indica que el patinete está acelerando. Una aceleración positiva significa que la velocidad del patinete está aumentando, lo que es coherente con el hecho de que el patinete comenzó desde el estado de inactividad y aumentó su velocidad al acelerar. Por lo tanto, la justificación del signo del resultado es que el patinete está en movimiento acelerado, saliendo del estado de reposo y aumentando su velocidad a lo largo del tiempo.